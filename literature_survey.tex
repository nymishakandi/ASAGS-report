\chapter{Literature Survey}
\begin{table}
\begin{center}
\begin{tabular}{|p{3cm}|p{3cm}|p{2cm}|p{4cm}|p{3cm}|}
\hline
\textbf{Title} & \textbf{Methodology} & \textbf{Results} & \textbf{Merits}  & \textbf{Demerits}\\
\hline
Violence detection using Oriented Violent Flows, 2016 [1] & AdaBoost and SVM classifier. & 88.00 percent & Feature representation model, which depicts the information involving both the motion magnitude and motion orientation. & Detection point where the behaviour is changing from normal to abnormal is time consuming.\\
\hline
Violent Flows:Real-Time Detection of Violent Crowd Behaviour, 2012 [2] & Global descriptors and SVM classifier & 5-fold cross validation: 81.30 percent & The algorithm detected far more violent scenes correctly, compared to existing work. It was furthermore far faster to detect the violence, typically in less than a second from its outbreak & Only magnitude of the flow vectors is considered, but the direction is not.\\
\hline

Automatic Fight
Detection in
Surveillance
Videos, 2016 [3] & 
Motion magnitude,
motion acceleration
and strength of
motion region
relationship,
collectively known as
motion signals & 
10 fold cross
validation:
82.70 percent & 
Difference
between
stimulated
fights and real
fights.
Doesn’t rely
on high level
behaviour
recognition,
Thus applicable to Low quality videos. & 
 Less accuracy is
achieved when
testing with real
fight scenarios\\
\hline

Online real-time
crowd behaviour
detection in video
sequences, 2015 [4] & 
Instant entropy and
temporal occupancy
variation & 
96 percent
Works
without the
need of
training phase. &
Computational
speed (FPS) is
varying for
different
datasets.\\
\hline
\end{tabular}
\end{center}
\caption{Literature Survey}
\end{table}