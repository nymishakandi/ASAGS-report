\chapter{Requirements, Analysis and Specifications}
\section{Unix}
Unix is a family of multitasking, multiuser computer operating systems that derive from the original AT\&T Unix, development starting in the 1970s at the Bell Labs research center by Ken Thompson, Dennis Ritchie, and others.
Unix was originally meant to be a basic platform for programmers developing software to be run on it and on other systems, rather than others. The system grew larger as the operating system started spreading in academic circles, as users added their own tools to the system and shared them with colleagues. The best feature of linux is that it is open source. Being open source makes anyone in the world to contribute to it. This makes unix highly robust and stable.  
\begin{itemize}
	\item Kernel - /usr/sys which contains the sub-components.
	\item Development Env - Linux code itself can be built in the linux. Provdes great environment for the programmers.
	\item Commands in Linux - \begin{itemize}
	\item  for system operation and maintenance, \item commands of general utility\item general-purpose applications  \item text formatting and \item typesetting package. 
	\item Document formatting - Unix systems were used  for document preparation and typesetting systems, and included many related programs such as nroff, troff, tbl, eqn, refer, and pic. Some modern Unix systems also include packages such as TeX and Ghostscript.
	\item Graphics - the plot subsystem provided facilities for producing simple vector plots in a device-independent format, with device-specific interpreters to display such files. Modern Unix systems also generally include X11 as a standard windowing system and GUI, and many support OpenGL.
	\item Communications - early Unix systems contained no inter-system communication, but did include the inter-user communication programs mail and write. V7 introduced the early inter-system communication system UUCP, and systems beginning with BSD release 4.1c included TCP/IP utilities.
\end{itemize}
\item Multiple Development Environments
\item Multithreaded Programming
\item Manipulation of Device Drivers
\item Support for latest languages and packages
\end{itemize}
\section{Python}
Only open source languages and tools are used in developing the project. Unix environment is being used for the development of the process.
Language used is Python and the main tool used is OpenCV for video processing. Python is interpreted language which is used for general-purpose programming. It is a user-friendly language which emphasizes on code readability and its syntax allows users to write programs with relatively fewer lines of code when compared to C/C++, Java etc. Python 2.7 version is used in the project.\\
Python's \textbf{features} include:
\begin{itemize}
\item East Learning: Python has less keywords, simple structure, and a clearly defined syntax. This allows to pick up the language quickly.
\item Scripting Language: Python code is more clearly defined and visible to the eyes. Since it is scripting language it can be easily read.
\item Maintenance: Python code is fairly easy-to-maintain.
\item A broad collection of standard library: Python's bulk of the library is very portable and cross-platform compatible on UNIX, Windows, and Macintosh. It uses the pip package for maintenance and installation of those standard packages.
\item Interactive Mode: Python has support for an interactive mode which allows interactive testing and debugging of snippets of code through tools such as iPython notebooks.
\item Portability: Python can run on a wide variety of hardware and has the same interpretation on all platforms.
\item Extendable Interpretation: You can add low-level modules to the Python interpreter. These modules enable programmers to add to or customize their tools to be more efficient.
\item Database Support: Python provides interfaces to all commercial databases such as Oracle, MySQL.
\item GUI Programming: Python supports GUI applications that can be created and ported to many system calls, libraries and windows systems, such as Windows MFC, Macintosh, and the X Window system of Unix.
\item Scalability: Python programs are highly scalable. They provide better structure and support than Shell Scripting.
\end{itemize}
Apart from the above-mentioned features, Python has a big list of good features, few are listed below:
\begin{itemize}
	\item It supports functional and structured programming methods as well as OOP.
\item It can be used as a scripting language or can be compiled to byte-code for building large applications.
\item It provides very high-level dynamic data types and supports dynamic type checking.
\item It supports automatic garbage collection.
\item It can be easily integrated with C, C++, COM, ActiveX, CORBA, and Java.

\end{itemize}
\section{OpenCV}
OpenCV (Open Source Computer Vision Library) is an open source computer vision and machine learning software library. OpenCV was built to provide a common infrastructure for computer vision applications and to accelerate the use of machine perception in the commercial products. Being a BSD-licensed product, OpenCV makes it easy for businesses to utilize and modify the code.\par
The library has more than 2500 optimized algorithms, which includes a comprehensive set of both classic and state-of-the-art computer vision and machine learning algorithms. These algorithms can be used to detect and recognize faces, identify objects, classify human actions in videos, track camera movements, track moving objects, extract 3D models of objects, produce 3D point clouds from stereo cameras, stitch images together to produce a high resolution image of an entire scene, find similar images from an image database, remove red eyes from images taken using flash, follow eye movements, recognize scenery and establish markers to overlay it with augmented reality, etc. OpenCV has more than 47 thousand people of user community and estimated number of downloads exceeding 14 million. The library is used extensively in companies, research groups and by governmental bodies.\par
Along with well-established companies like Google, Yahoo, Microsoft, Intel, IBM, Sony, Honda, Toyota that employ the library, there are many start-ups such as Applied Minds, VideoSurf, and Zeitera, that make extensive use of OpenCV. OpenCV’s deployed uses span the range from stitching street view images together, detecting intrusions in surveillance video in Israel, monitoring mine equipment in China, helping robots navigate and pick up objects at Willow Garage, detection of swimming pool drowning accidents in Europe, running interactive art in Spain and New York, checking runways for debris in Turkey, inspecting labels on products in factories around the world on to rapid face detection in Japan.\par
OpenCV is written in C++ and its primary interface is in C++, but it still retains a less comprehensive though extensive older C interface. There are bindings in Python, Java and MATLAB/OCTAVE. The API for these interfaces	can be found in the online documentation. Wrappers in other languages such as C-sharp, Perl, Ch, Haskell and Ruby have been developed to encourage adoption by a wider audience. All the new developments and algorithms in OpenCV are now developed in the C++ interface.\\
There are many \textbf{applications} of OpenCV. A few of them are cited as below:
\begin{itemize}
\item 2D and 3D feature toolkits 
\item Facial recognition system 
\item Gesture recognition 
\item Human Computer Interaction (HCI) 
\item Mobile robotics 
\item Motion understanding 
\item Object identification 
\item Segmentation and recognition 
\item Stereopsis stereo vision: depth perception from 2 cameras 
\item Structure From Motion (SFM) 
\item Motion tracking 
\end{itemize}
\section{Hardware Requirements}
\begin{itemize}
	\item Processor - Intel Core i5 5th gen
	\item RAM - 8 GB
	\item Hard Disk - 500 GB
	\item OS - Unix based such as Debian , MacOS
	\item WebCam
	\item USB 2.0 Ports
\end{itemize}
\section{Keras and TensorFlow}
\subsection{Keras}
Keras is a high-level neural networks API, written in Python and capable of running on top of \textbf{TensorFlow, CNTK, or Theano}. It was developed with a focus on enabling fast experimentation. Being able to go from idea to result with the least possible delay is key to doing good research.
\\
Use Keras if you need a deep learning library that:
\begin{itemize}
	\item Allows for easy and fast prototyping (through user friendliness, modularity, and extensibility).
	\item Supports both convolutional networks and recurrent networks, as well as combinations of the two.
	\item Runs seamlessly on CPU and GPU.
\end{itemize}
Keras is compatible with: \textbf{Python 2.7-3.6.}
\\
\textbf{Guiding Principles:}
\begin{itemize}
	\item \textbf{User friendliness}. Keras is an API designed for human beings, not machines. It puts user experience front and center. Keras follows best practices for reducing cognitive load: it offers consistent \& simple APIs, it minimizes the number of user actions required for common use cases, and it provides clear and actionable feedback upon user error.
	\item \textbf{Modularity}.A model is understood as a sequence or a graph of standalone, fully-configurable modules that can be plugged together with as little restrictions as possible. In particular, neural layers, cost functions, optimizers, initialization schemes, activation functions, regularization schemes are all standalone modules that you can combine to create new models.
	\item \textbf{Easy extensibility}. New modules are simple to add (as new classes and functions), and existing modules provide ample examples. To be able to easily create new modules allows for total expressiveness, making Keras suitable for advanced research.
	\item \textbf{Work with Python}. No separate models configuration files in a declarative format. Models are described in Python code, which is compact, easier to debug, and allows for ease of extensibility.
\end{itemize}
\subsection{TensorFlow}
TensorFlow is an open-source software library for dataflow programming across a range of tasks. It is a symbolic math library, and also used for machine learning applications such as neural networks.
\par
TensorFlow is an open source software library for numerical computation using data flow graphs. The graph nodes represent mathematical operations, while the graph edges represent the multidimensional data arrays (tensors) that flow between them. This flexible architecture lets you deploy computation to one or more CPUs or GPUs in a desktop, server, or mobile device without rewriting code. TensorFlow also includes TensorBoard, a data visualization toolkit.
\par
TensorFlow was originally developed by researchers and engineers working on the Google Brain team within Google's Machine Intelligence Research organization for the purposes of conducting machine learning and deep neural networks research. The system is general enough to be applicable in a wide variety of other domains, as well.
\section{Others}
\subsection{Libraries}
Along with OpenCV we have used bob. Bob is a machine learning platform used in python, it helped us to port Matlab code of optical flow to python. Scikit learn is Python package which is similar to WEKA tool for java. It will be used for training a classifier model in the proposed algorithm of the project.

\subsection{Editor}
An editor is required to create python source files and to view violent features generated for the dataset. Open source editors like Atom and PyCharm have been used which also provide terminal access.

\subsection{Video Player}
To play all the videos through openCV and to view some random videos. Video players like VLC is required which is open source. ffmpeg library is also required which provides a way to read video files through openCV. ffmpeg can also be used to format videos.