\chapter{Introduction}

\section{Problem Statement}
Automation in Real-Time analysis of crowd can make surveillance more efficient. In this modern era, the number of cameras for surveillance is continuously increasing which leads to increase in burden on the human. Real Time alert generation is a system that can analyse abnormalities accurately in real time and create an alert. These automatic alerts may help to proact rather than to react. 

\section{Objective}
Public safety is an important concern for any organization. So as to achieve that an organization takes the help of CCTV cameras. With the decrease in price of cameras, amount of information generated in terms of cctv footage is huge. Real time processing of this footage is required so that the burden on the human surveyors may be decreased.
\par
Aim of this project is to implement an algorithm that can process the incoming footage in real time and detect disturbance within few seconds of an abnormal activity. Crucial amount of time will be saved if an alert is generated. This time may be the difference between the life and death of a person.

\section{Motivation}
Crowd can be defined as a large number of people in close proximity to each other. Whenever an abnormal event happens in crowd, all the people present in the crowd react to that at once. This gives us opportunity to  detect violence in crowd, if we detect that exact moment where the abnormal activity happens.
\par 
If an alert is generated during the exact moment when the outburst takes place, it may be used to alert the local bodies such as riot control team to take control of the situation. It would be highly beneficial for us to detect violence at the moment it occurs rather than reacting to the incident later on.

\section{Existing System}
Now-a-days every public area will have CCTV coverage so as to protect the public. In the existing manual surveillance system, a human surveyor continuously pays attention to screens. As the number of cameras increase burden on the human will also increase. This system is laggy and it may or may not detect every outbreak. 
\par
Violence detection is a part of “Action Recognition”. There has been intensive research on action recognition in the past. Much research has been done in person to person fight detection, sports violence detection, violence detection in movies and slow motion fight detection. Violence detection in crowd is one of the most trending topics in the area of action recognition. 
\par
Existing Person-to-Person fight detection takes heavy computational power and cannot be refined to be used in real-time detection. Blob method used is quick and requires less computational power but it can be used for only Person-to-Person fight detection. We cannot refine it to be used for crowd violence detection.
\subsection{Problems in Existing System}
\begin{itemize}
	\item As the number of cameras increase burden on the human will also increase. This system is laggy and it may or may not detect every outbreak.
	\item Existing Person-to-Person fight detection takes heavy computational power and cannot be refined to be used in real-time detection.
	\item Blob method used is quick and requires less computational power but it can be used for only Person-to-Person fight detection, we cannot refine it to be used for crowd violence detection.
\end{itemize}
\section{Proposed System}
\begin{itemize}
  \item We propose an automated system to detect and generate alert in real time incase outbreak of violence in crowd using Video Processing, Optical Flow and Violent Flow Descriptors(ViF). 
  \item In the proposed system, surveillance videos are taken as input and output is detection of violence(if present) in the video.
  \item Our system works in real time i.e it detects disturbance or violence in crowd present in the video within milliseconds of outset of violence. 
  \item The Real time detection of violence helps to proact rather than to react.
\end{itemize}
\section{Organisation of Project Report}
This project report is mainly divided into 6 modules as follows:
\begin{itemize}
  \item The second chapter deals with the Introduction of the Project and explanation about some basic concepts.
  \item The third chapter discusses about the literature survey of this project which includes an insight into the core concepts of our project.
  \item Fourth chapter deals with the Hardware and Software requirements of the project.
  \item Fifth chapter gives the methodology of the project, the way proposed algorithm is generated.
  \item Sixth chapter gives details of the "in the wild" dataset.
  \item The seventh chapter deals with the design of our proposed system. 
  \item The eighth chapter deals with the implementation of our system which discusses about the algorithms used in building our system.
  \item The ninth chapter displays our results and discussions through a series of screenshots. 
  \item From tenth chapter onwards details about the conclusions, future scope of our project and the limitations. 
\end{itemize}


